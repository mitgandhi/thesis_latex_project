\course{Computer Science}
\matriculationyear{2020}
\issuedate{26.2.2025}
\duedate{31.07.2025}
\professor{Prof. Dr.-Ing. Jürgen Weber}

\begin{task}[Enhancing Operational Speed Limits of Axial Piston Pumps] % Custom title provided
    \minisec{\objectivesname}\smallskip

    Axial piston pumps are crucial in numerous hydraulic systems, where high pressure and efficiency
    are paramount. As the transition to electrification accelerates, these pumps must evolve to operate
    efficiently at higher speeds, aligning with the performance characteristics of electric motors, which
    typically deliver optimal output at elevated rotational speeds.
    
    A significant challenge at these higher operational speeds is managing cavitation and lubricating gaps. 
    At increased velocities, pistons may tilt excessively, risking the occurrence of the piston-stick 
    phenomenon—an abrupt and destructive failure mode that can cause catastrophic pump damage.
    
    This thesis will focus on exploring, modeling, and assessing various strategies to enhance the
    operational speed limits of piston pumps – with focus on pistons. These strategies may include
    implementing inclined geometries and advanced surface structures designed to mitigate piston tilt
    and promote stable operation under high-speed conditions. By leveraging and modifying an existing
    simulation tool, Caspar FSTI, this research will introduce piston tilt dynamics to evaluate the
    effectiveness of these modifications and compare various measures based on effectiveness and
    ease of implementation.
    
    The sub tasks include:
    \begin{itemize}
        \item Literature review with focus on tribological bearings specifically for high-speed
              applications and swashplate pumps
        \item Implementation of inclined pistons in Caspar FSTI code
        \item Choose reference pump and simulation model finding its current maximum speed
        \item Validate with test results such as wear spots, leakages, etc.
        \item Optimize gap for high speed based on gap length and clearance using a Machine
              Learning Model to reduce simulation effort
        \item Find optimal piston dimensions for inclined pistons and/or surface structures by
              expanding the ML Model
        \item Compare and rate various speed enhancing measures
        \item Generalize findings and transfer optimal design to another pump
        \item Documentation
    \end{itemize}
    
    \minisec{\focusname}\smallskip
    \begin{itemize}
        \item Research \& Analysis
        \item Development of a concept \& Application of the developed methodology
        \item Documentation and graphical presentation of the results
    \end{itemize}
\end{task}