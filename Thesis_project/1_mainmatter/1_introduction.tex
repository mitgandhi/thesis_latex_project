\iflanguage{ngerman}
%{\chapter{Einleitung}}
{\chapter{Introduction}}

\label{sec:introduction}
Across numerous industrial sectors—from aerospace to robotics—compact hydraulic systems with exceptional power density have become essential components. The conversion between mechanical and hydraulic energy occurs bidirectionally: machines function as pumps when transforming mechanical energy into hydraulic power and as motors when performing the reverse process.

\p Among the various displacement designs available (including gear, vane, and screw configurations), axial piston machines with swashplate architecture stand out for high-pressure applications. Their notable efficiency and control advantages come with inherent design complexities that affect manufacturing costs.

\p The expanding implementation of these systems, particularly with emerging displacement control technologies, creates market pressure for cost-effective yet high-performance units. Meeting these demands requires transformative development approaches, particularly through digital simulation models. This computational prototyping methodology provides deeper insights into internal physical processes, enabling engineers to create more reliable and efficient units while reducing development timeframes and expenses.
\section{Motivation}
Here you motivate, why you are doing your research.

\section{Goal}
This section should clarify, what should be achieved by the work.

\section{Structure of the Work}
Here the structure can be \emph{briefly} explained.


